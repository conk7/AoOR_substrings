\documentclass{article}

\usepackage[left=10mm,right=10mm,top=10mm,bottom=20mm,paper=a4paper]{geometry}

% межстрочный интервал
\linespread{1.5}

% top page pagination
\usepackage{fancyhdr}
\fancyhf{}
\fancyhead[C]{\thepage}
\renewcommand\headrulewidth{0pt}

% times new roman support
\usepackage{fontspec}
\setmainfont{Times New Roman}

% more font sizes
\usepackage{scrextend}

% Formatting TOC
\usepackage{tocloft}
\renewcommand{\cftsecfont}{\normalfont\mdseries}% titles in non-bold
\renewcommand{\cftsecpagefont}{\normalfont\mdseries}% page numbers in non-bold
\renewcommand{\cftsecleader}{\mdseries\cftdotfill{\cftdotsep}}% dot leaders in non-bold
\renewcommand{\contentsname}{}
\renewcommand{\cftsecaftersnum}{.}
\renewcommand{\cftsubsecaftersnum}{.}
\usepackage{titlesec}
\titleformat{\section}{\normalfont\center\bfseries\uppercase}{}{0em}{}
\titleformat{\subsection}{\normalfont\center\bfseries}{\thesubsection}{0.5em}{}

% remove footnote margin
\usepackage{footmisc}
\usepackage{url}
\urlstyle{same}

% tables support
\usepackage{float}
\usepackage{multirow}
\usepackage{tabularx}
\usepackage{tablefootnote}
\usepackage{caption} 
% custom captions
\renewcommand{\figurename}{Рис.}
\renewcommand{\tablename}{Таблица}
\captionsetup[table]{labelsep=endash,justification=centering,singlelinecheck=false,font=normalsize}

% image support
\usepackage{graphicx}
\graphicspath{ {images/} }

% math symbols (R, C, N)
\usepackage{amssymb}


\title{CourseWork}
\author{}
\date{}

\begin{document}

% абзацный отступ
\setlength{\parindent}{35pt}
% remove footnote margin
\setlength{\footnotemargin}{5pt}
% text justification
\sloppy
\frenchspacing % fixes double spaces after commas

\KOMAoption{fontsize}{14pt}


\begin{table}[!h]
    \begin{center}
        \begin{tabular}{ | m{4.1em} | m{4.5em} | m{4.5em} | m{4.5em} | m{4.5em} | m{4.5em} | m{4.5em}| }
            \hline
            Бенчмарк   & \multicolumn{2}{p{9em}|}{Наивный алгоритм} & \multicolumn{2}{p{9em}|}{Алгоритм Бойера-Мура-Хорспула} & \multicolumn{2}{p{9em}|}{Алгоритм Кнутта-Мориса-Пратта}                                     \\
            \hline
                       & Время(мс)                                  & Сравнения                                               & Время(мс)                                               & Сравнения & Время(мс) & Сравнения \\
            \hline
            bad\_t\_1  & 0.0015                                     & 18                                                      & 0.0014                                                  & 10        & 0.0021    & 19        \\
            \hline
            bad\_t\_2  & 0.0437                                     & 910                                                     & 0.0098                                                  & 100       & 0.0154    & 207       \\
            \hline
            bad\_t\_3  & 4.6323                                     & 90100                                                   & 0.1287                                                  & 1000      & 0.1814    & 2097      \\
            \hline
            bad\_t\_4  & 240.9398                                   & 4001000                                                 & 0.7371                                                  & 5000      & 1.1571    & 10997     \\
            \hline
            good\_t\_1 & 0.1041                                     & 714                                                     & 0.0152                                                  & 76        & 0.1027    & 730       \\
            \hline
            good\_t\_2 & 0.1672                                     & 1158                                                    & 0.0221                                                  & 122       & 0.1679    & 1242      \\
            \hline
            good\_t\_3 & 0.5980                                     & 3554                                                    & 0.0943                                                  & 472       & 0.6008    & 3932      \\
            \hline
            good\_t\_4 & 1.6686                                     & 10714                                                   & 0.1126                                                  & 554       & 1.5480    & 10805     \\
            \hline
        \end{tabular}
        \caption{Результаты бенчмарков.}
    \end{center}
\end{table}

В таблице 1 приведены результаты бенчмарков каждого алгоритма.
Данные результаты получены на основе усреднения результатов 100 запусков каждого бенчмарка.
В тексте bad\_t\_1 искался паттерн bad\_w\_1, аналогично со всеми остальными файлами.

Рассмотрим плохие случаи. Наивный алгоритм значительно уступает другим как по времени,
так и по операциям сравнения.
Лучше всего в данных бенчмарках себя проявил алгоритм
Бойера-Мура-Хорспула (БМХ). Он продемонстрировал кратно меньшее время и
количество операций сравнения по сравнению с наивным.
Алгоритм Кнутта-Мориса-Пратта (КМП) слегка хуже алгоритма БМХ,
так как процедура построения таблицы сдвигов у него сложнее. Несмотря на это,
данный алгоритм все равно отрабатывает значительно быстрее наивного.

Перейдем к хорошим случаям. В них тоже лидирует алгоритм БМХ. Отставание алгоритма КМП
от алгоритма БМХ значительно увеличилось, поскольку в данных возросла мощность алфавитов.
Наивный алгоритм, в свою очередь,
снова оказался на последнем месте, показав слегка худшие результаты,
чем алгоритм КМП.

В заключение можно сказать, что алгоритм БМХ является лучшим среди представленных.


\end{document}
